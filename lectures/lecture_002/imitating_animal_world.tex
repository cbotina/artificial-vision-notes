\subsection{Imitating the Animal World}

There are multiple ways to describe perception systems. In this topic, we follow an approach that goes from the most generic to the most concrete. Perception systems attempt to imitate nature, so it is logical to start by considering certain simplifications.

Without ceasing to imitate nature, we will model systems that reflect the behavior of simple organisms, such as individuals that do not even possess eyes or specific elements for visual information capture, like mollusks. These beings perform three very simple perception functions:

\subsubsection{Information Capture}

This process consists of obtaining external stimuli that reflect what activity (movements, temperature changes, threats, etc.) is taking place. Information capture can consider multiple stimuli, mainly physical, mechanical, or chemical. The information collected is normally greater than that needed to understand the exterior. The adaptation of this excess sensitivity to environmental needs is part of evolution itself.

\subsubsection{Processing}

Captured information needs subsequent processing to eliminate unnecessary and redundant data, understand if it is sufficient, and otherwise redirect information capture toward another point in space or combine information from a given source with other sources or past information. Without this processing, a living being could draw erroneous conclusions, confuse events, or even lead to its extinction.

An essential characteristic of this processing is that it must, at all times, be performed at the same speed at which the initial data are being captured. This ensures that only the necessary information reaches the brain. Conversely, an animal that stores everything it has heard during the day and analyzes it at night will not be able to dodge the instant attack of a predator.

\subsubsection{Decision Making and Learning}

The main purpose of perceiving the exterior is decision making. Better decision making (based on more information, context, or diversity of sources) will undoubtedly lead to a greater competitive advantage. This decision making has a direct consequence: learning. The living being will learn that a certain stimulus associated with a certain decision will have consequences. These consequences will be stored to optimize both decision making and information capture.
